%% Sample file $Id: PBML_article.tex 92 2007-12-18 22:59:46Z zw $
% 
% This is a sample file for the PBML article.
% 
% The Prague Bulletin of Mathematical Linguistics is typeset by XeLaTeX. This
% means that it is not guaranteed that the line and page breaks will come up
% exactly the same in the final printed version as in your computer. It is
% not required that you use XeLaTeX and the same fonts as will be used in the
% journal but it is better to do so if you could. You can even used standard
% LaTeX if XeLaTeX is not available on your computer.
% 
% 
% Document class
% ==============
% 
% The Prague Bulletin of Mathematical Linguistics uses its own document class.
% You will thus start your document with:

\documentclass{pbml}

% If you process the document with XeLaTeX, fonts Minion Pro (distributed
% with Adobe Reader) and DejaVu
% (http://dejavu.sourceforge.net/wiki/index.php/Main_Page). If you do not
% have these fonts and are not able to install them, you can instruct XeLaTeX
% to use its default fonts by issuing \document[nofonts]{pbml}.
% 
% 
% Packages
% ========
% 
% I this place you will load required packages by the \usepackage commands.
% Remember that other articles may require other packages and it is known
% that some of them are in conflict. Please use only those packages that you
% really need.
% 
% The following packages are loaded automatically by the class:
% euler (for math fonts)
% graphicx (for inclusion of images)
% multicol (for multicolumn typesetting)
% fullname (for bibliography citations)
% 
% If XeLaTeX is used, fontspec and xltxtra are loaded as well.
% 
% If standard LaTeX is used, the font encoding is switched to T1 and Latin
% Modern fonts are used if they are found in your TeX distribution.
% 
% 
% Graphics
% ========
% 
% The most portable graphics package is tikz which is a front end to pgf.
% This package is fully supported both in standard LaTeX and XeLaTeX.
% PSTricks is not currently available for XeLaTeX. It will most probably be
% supported in the future.
% 
% 
% Fonts
% =====
% 
% If you want to typset examples in east Asian scripts, you have to use
% OpenType Unicode fonts that are freely redistributable and you have to
% include them with your article. If you must use nonfree or non-Unicode
% fonts, you must supply the examples as EPS or PDF with fonts embedded in
% the files (as the required subset).
% 
% 
% Definitions
% ===========
% 
% You can use your own macros defined by \newcommand, \providecommand,
% \DeclareRobustCommand (or even by \def) as well as environments declared by
% \newenvironment. These definitions will automatically be local to your
% article and will not clash with other parts of the journal. You can also
% use \newcounter in order to declare your own counters.
% 
% 
% Beginning of the document
% =========================
% 
% Use the standard command:

\begin{document}


% Document title
% ==============
% 
% Due to journal organization the article title and authors must be specified
% AFTER \begin{document}. Give the article title:

\title{Title of My Article}

% You may optionally specify a subtitle:

\subtitle{Optional Subtitle}

% Now put the author names and adresses, one command for each authors. The
% order of authors printed below the title will be the same as the order of
% your commands:

\author{firstname=Humpty, surname=Dumpty,
       address={High Wall, Looking Glass}}
\author{firstname=Mock, surname=Turtle, address=Sea Coast\\Wonderland}
\author{firstname=Cheshire, surname=Cat,
        address=Duchess Palace\\Wonderland}

% The braces are mandatory only if the value contains a comma as in the first
% author's address but they never make any harm, you can safely put all
% values into braces.
% 
% The title and authors' names are used in the running head. If they are
% long, you should define short versions. These definitions are optional. You
% define them only if they are needed. The example follows:

\shorttitle{Short}
\shortauthor{H. Dumpty, M. Turtle, C. Cat}

% Now print the title by:

\maketitle


% Abstract
% ========
% 
% The abstract is placed within the "abstract" environment. It is a mandatory
% part of the article.

\begin{abstract}
The abstract of the article...
\end{abstract}


% The body of the article
% =======================
% 
% The PBML class is modelled after the standard article class. This means
% that you can use almost everything that is allowed in articles as described
% in the textbooks of LaTeX. We support sectioning commands \section,
% \subsection and \subsubsection. In addition to standard LaTeX we support
% \subsection and \subsubsection with an empty title (i.e. just with empty
% braces {} that must not even contain a space). Such subsections will be
% automatically numbered and properly formatted.
% 
% 
% Figures and tables
% ==================
% 
% Use figure and table environments. The \caption command is redefined in
% order to comply with the graphic design.
% 
% 
% Inclusion of images and other files
% ===================================
% 
% All files required for your articles must reside in the same directory. You
% are allowed to use subdirectories only exceptionally. When loading the file
% you must use a relative path, never the full path. Load the images by
% \includegraphics and other files by \input, never use \include.
% 
% 
% Cross references
% ================
% 
% You can safely used \label, \ref and \pageref with any identifiers. These
% macros are modified in order not to collide with identifiers in other
% articles.
% 
% 
% Bibliography
% ============
% 
% You may either enter the bibliography manually, possibly making use of
% \label and \ref, or use BibTeX. The bibliography style is set
% automatically. The \cite command is also modified so that it does not
% collide with other articles. You process the bibliography by BibTeX in the
% standard way and include it by:

\bibliography{mybib}


% End of the article
% ==================
% 
% The article must end with:

\end{document}
