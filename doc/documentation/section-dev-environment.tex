\section{Development Environment}

Joshua is written primarily in Java. The code is designed so that any programmer familiar with Java and the relevant machine translation literature should be able to read, understand, and easily extend the code.

Any decent editor should be fine for Joshua development. Joshua is set up so that ant can be used to easily compile the code.

\subsection{Joshua in Eclipse}

Many Java developers find that using a development environment can be very helpful. Joshua can easily be used in Eclipse. To set up a Joshua development environment in Eclipse, first install the Subclipse SVN plugin in Eclipse:

\begin{verbatim}
File -> New -> Other

SVN -> Checkout project from SVN

Click Next

Select "Create a new repository location" 
Click Next
Url: https://joshua.svn.sourceforge.net/svnroot/joshua/trunk
Click Next

In the Select folder dialog, 
select the node for "https://joshua.svn.sourceforge.net/svnroot/joshua/trunk"
Click Next

In the dialog "Check out as a project configured using the New Project Wizard" 
should be selected.
Click Finish.

In the New Project dialog that pops up, select "Java Project".
Click Next.

In the New Java Project dialog, type in a Project name (ex: joshua).
Click Finish.
Click OK.

A popup will show "SVN Checkout".
You should now have the project checked out.

Right-click on the project folder, and select Properties.
Select Java Build Path.
Select the Libraries tab.
Click Add JARs.
All all jars in lib/
Click OK.
Click OK.

Go to the command line and cd to project/joshua/trunk.
Type ant

Now go back to Eclipse and refresh the view by right-clicking on the project and choosing Refresh.
Tell Eclipse to do a clean build by selecting Clean from the Project menu.

\end{verbatim}
