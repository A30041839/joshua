\chapter{Setting Up Joshua}

\section{How to get Joshua}
\label{sec:get-joshua}

Joshua can be obtained at \url{http://sf.net/projects/joshua}.

The best way to obtain a copy of the most recent Joshua source code is to check
out from the main SVN repository. For this, you need Subversion, which
is likely already present on your system. If not, if can be obtained
freely at \url{http://subversion.tigris.org}. If you already have a copy of
Joshua, please skip to the Prerequisites or Installing \& Compiling
sections of this document.

To set up your own copy of Joshua, you start by checking out the main
development branch:

\begin{verbatim}
$ svn co https://joshua.svn.sourceforge.net/svnroot/joshua/trunk joshua
\end{verbatim}

This creates a subdirectory with your local copy of the source
code. This directory is under version control, which means that your
local copy is tied to the main repository. This facilitates keeping
your code up-to-date as well as contributing your changes back to the
repository.

To fetch the latest changes, fixes or improvements to the decoder, you
simply run:

\begin{verbatim}
$ cd joshua
$ svn up
\end{verbatim}

Detailed documentation for subversion can be found in the \href{http://svnbook.red-bean.com}{Subversion Book}.


\section{Prerequisites}

\subsection{Java}

Joshua is written in Java, and thus requires a Java SDK to be
installed. Please make sure you use a recent version of Java. Make
sure you have \$JAVA\_HOME set to the SDK directory.

For Mac OS X this usually is done by adding

\begin{verbatim}
export JAVA_HOME="/Library/Java/Home"
\end{verbatim}

to your .bashrc, .bash\_profile or .profile file.

\subsection{SRILM}

SRILM is a widely used language modeling toolkit, available for
download at \url{http://www.speech.sri.com/projects/srilm}

Make sure you have the \$SRILM variable set to the directory you
installed SRILM in, i.e.

\begin{verbatim}
$ export SRILM=/path/to/srilm
\end{verbatim}

While SRILM is not strictly required, if you plan to run any real-sized experiments it will be necessary. Make sure that you have compiled and installed SRILM, and set the SRILM environment variable to the location where SRILM is installed.

\subsection{Swig}

Swig is an inter-language wrapper. It is required to use SRILM with Joshua. Swig can be obtained at \url{http://www.swig.org}. 

\subsection{Apache Ant}

Apache Ant is a Java building tool with functionality similar to the
make tool. It can be found at \url{http://ant.apache.org}. Use of ant is not required, but it is the suggested build tool for compiling Joshua.

      
\section{How to build Joshua}
\label{sec:build-joshua}

First of all, make sure that all prerequisites (Ant, Swig, SRILM, etc) are installed and properly accessible from the command line. SRILM should be built prior
to this step. For convenience, you may wish to set the JOSHUA\_HOME 
environment to the directory you installed Joshua in.

To build Joshua, it is sufficient to change into its install directory
and run make:

\begin{verbatim}
$ cd $JOSHUA_HOME
$ ant compile
\end{verbatim}

This builds the Java source code, as well as the SRILM wrapper. Similarly, 
if you have changes to the code, you can rebuild the decoder using the same 
command.

For a full rebuild of the decoder, simply run

\begin{verbatim}
$ ant clean
\end{verbatim}

before building. This command will remove any previously compiled code.


\section{How to run tests and examples}

Joshua comes with a suite of unit tests that help to verify correct behavior. These tests should be run prior to trying Joshua, as a sanity check to verify that the code is behaving as expected. To run the Joshua unit tests:

\begin{verbatim}
$ ant test
\end{verbatim}


In addition, some regression testing scripts are provided. These scripts also serve as examples of how to run the Joshua decoder. To run these scripts:

\begin{verbatim}
$ ./example/decode_example_javalm.sh
\end{verbatim}

or

\begin{verbatim}
$ ./example/decode_example_srilm.sh
\end{verbatim}



\section{How to train and run Joshua}
\label{sec:train-joshua}

How to quickly get Joshua up and running:

 \begin{enumerate}
 \item Train a language model
 \item Use the joshua.prefix\_tree.ExtractRules class to extract a grammar for a dev set, given an aligned parallel training corpus (see \cref{ch:grammar-extraction})
 \item Create (or copy from an example) a basic Joshua config file (see the example folder)
 \item Using that grammar, run joshua.zmert.ZMERT to perform minimum error rate training on the dev set. (see INSTALL.txt and doc/zmert\_release). This will result in a set of lambda parameter values.
\item Update your Joshua config file to use the new lambda parameter values.
\item Use the joshua.prefix\_tree.ExtractRules class to extract a grammar for a test set, given an aligned parallel training corpus (see \cref{ch:grammar-extraction})
\item Using the test set grammar and the updated config file, translate the test set using joshua.decoder.JoshuaDecoder (see INSTALL.txt and scripts in example/)
 \end{enumerate}


\section{How to package Joshua}

Some users may be interested in packaging Joshua into a single distributable file. To pack the decoder into a JAR archive, either compiled or the source code, run

\begin{verbatim}
$ ant jar
\end{verbatim}

or

\begin{verbatim}
$ ant source-jar
\end{verbatim}
