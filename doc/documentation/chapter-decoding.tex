\chapter{Decoding}

Assumption: The grammar contains no epsilon rules.

\begin{enumerate}
\item Extend the dot chart
\item Populate complete rules into the chart, as in regular CKY
\item Process unary rules (e.g., S $\rightarrow$ X, NP $\rightarrow$ NN), just add these items in chart, assume the rule set is acyclic
\item 
\end{enumerate}


\section{Data structures}

At the lowest level, Joshua grammars are stored internally in a trie data structure. This structure is specified by the {\tt Trie} interface in the {\tt joshua.decoder.ff.tm} package.

The decoding algorithm makes use of cube pruning \cite{chiang07}. Cube pruning requires that all rules which share a common source language right-hand side be sorted. The {\tt SortableGrammar} interface in the {\tt joshua.decoder.ff.tm} package provides a specification based on this requirement. A key method of {\tt SortableGrammar} is {\tt getTrieRoot()}, which returns the root node of the trie that defines the grammar.

The abstract class {\tt AbstractGrammar} in the {\tt joshua.decoder.ff.tm} package provides a default implementation of the sorting mechanism specified by {\tt SortableGrammar}. All concrete grammar implementations should inherit from {\tt AbstractGrammar}.


\section{Extend the dot chart}

Attempt to combine an item in the dot chart
with an item in the chart to create a new item
in the dot chart.

In other words, this method looks 
for (proved) theorems or axioms in the completed chart
that may apply and extend the dot position.

You can think of an item in the dot chart as a pointer into the grammar trie, plus some history data.


\section{Populate completed chart}

The grammar is assumed to be stored as a trie data structure. Each node in the trie may contain a list of translation rules.

Populate complete rules into the chart, as in regular CKY

Pruning, feature function evaluation, and creation of hypergraph all take place in this step



\section{Process unary rules}

Process unary rules (e.g., S $\rightarrow$ X, NP $\rightarrow$ NN), just add these items in chart, assume the rule set is acyclic.

This step uses an agenda parsing approach.


\section{}

